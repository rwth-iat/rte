%	HeaderVorlesung.tex
%	Version 041013_060104
%

%	--------------------------------------------------------------------------------

%	Pakete
%
\usepackage[latin1]{inputenc}%
\usepackage[T1]{fontenc}%
\usepackage{scrpage2}%
\usepackage{times}%
\usepackage{graphicx}%

%	- - - - - - - - - - - - - - - - - - - - - - - - - - - - - - - - - - - - - - - - 

%	Titelseite
%
\newcommand{\Titelseite}[4]{%
	%\frontmatter%
	\begin{titlepage}%
	\begin{flushright}%
	\vspace*{2.5cm}%
	{\Huge\strut%
	#1%
	}\\%
	{\LARGE\strut%
	#2%
	}\\\rule{\textwidth}{1mm}\\%
	\vskip 1mm%
	#3\\%
	#4%
	\vskip 1cm%
	Andreas Sch\"uller%
	\vfill%
	{\small%
	Lehrstuhl f�r Prozessleittechnik\\%
	Prof. Dr.-Ing. Ulrich Epple\\%
	RWTH Aachen\\%
	D-52064 Aachen, Deutschland\\%
	Telefon +49 (0) 241 80 94339\\%
	Fax +49 (0) 241 80 92238\\%
	www.plt.rwth-aachen.de%
	}%
	\end{flushright}%
	\end{titlepage}%
}%

%	- - - - - - - - - - - - - - - - - - - - - - - - - - - - - - - - - - - - - - - - 

%	Seitenstil / Kopf- und Fusszeile
%
\newcommand{\PageStyle}[0]{%	
	\pagestyle{scrheadings}%
	\clearscrheadings%
	\clearscrplain%
	\clearscrheadfoot%
	\setheadsepline{.5pt}%
	%\setfootsepline{.5pt}%
	\ihead[]{}%
	\ohead[]{\headmark}%
	\ifoot[Lehrstuhl f�r Prozessleittechnik]{Lehrstuhl f�r Prozessleittechnik}%
	\ofoot[\pagemark]{\pagemark}%
}%
	
%	--------------------------------------------------------------------------------
	
%	jpg-Abbildung mit Bildunterschrift (Caption)
%		#1: Dateiname des Bildes
%		#2: Ausrichtung
%		#3: gew�nschte Breite
%		#4: Referenz
%		#5: Bildunterschrift
%		#6: Anordnung auf der Seite (optionales "p")
%
\newcommand{\jpgCap}[6]{%
	\begin{figure}[htb#6]%
		\begin{#2}%
			\includegraphics[width=#3]{../Bilder/#1.jpg}%
		\end{#2}%
		\caption{#5}%
		\label{fig:#4}%
	\end{figure}%
}%

\newcommand{\pdfCap}[6]{%
	\begin{figure}[htb#6]%
		\begin{#2}%
			\includegraphics[width=#3]{../Bilder/#1.pdf}%
		\end{#2}%
		\caption{#5}%
		\label{fig:#4}%
	\end{figure}%
}%

%	- - - - - - - - - - - - - - - - - - - - - - - - - - - - - - - - - - - - - - - - 	
	
%	Referenz auf jpg-Abbildung
%
\newcommand{\jpgRef}[1]{%
	Abbildung \ref{fig:#1}%
}%

%	- - - - - - - - - - - - - - - - - - - - - - - - - - - - - - - - - - - - - - - - 

%	Schriftstil: fett
%
\newcommand{\fett}[1]{%
	\textbf{#1}%
}%

%	- - - - - - - - - - - - - - - - - - - - - - - - - - - - - - - - - - - - - - - - 

%	Schriftstil: kursiv
%
\newcommand{\kursiv}[1]{%
	\textit{#1}%
}%

%	- - - - - - - - - - - - - - - - - - - - - - - - - - - - - - - - - - - - - - - - 

%	Schriftstil: Kapitaelchen
%
\newcommand{\kapitaelchen}[1]{%
	\textsc{#1}%
}%

%	- - - - - - - - - - - - - - - - - - - - - - - - - - - - - - - - - - - - - - - - 