\chapter{Einleitung}

Die Bibliothek \textbf{iec61131stdfb} stellt eine Reihe standardisierter Funktionsbausteine zur Verfügung. Diese Bausteine sind kompatibel zur gleichnamigen Norm (IEC61131-3 Stand 28.10.2010) implementiert. Unter den Bausteinen sind einfache Rechenoperationen, trigonometrische Funktionen, Potenzieren zur Euler'schen Zahl oder einer beliebigen Basis, natürlicher und 10er-Logarithmus vorhanden. Des Weiteren gibt es Bausteine zum Vergleichen von Werten, zu deren Demultiplexen oder Auswählen nach Maximum oder Minimum und zur Begrenzung eines Wertes nach oben und unten. Boolsche Logiken sind ebenso vorhanden. Außerdem gibt es Flankenerkennungen, FlipFlops, Zähler und Timer-Bausteine.\\
Mit dieser Auswahl an Bausteinen lassen sich die nahezu alle Funktionen im Leitsystem umsetzen. Zusätzliche Bausteine wie Integratoren, Differetiatoren und Regler sowie Totzeitglieder werden zukünftig noch hinzugefügt.\\

Dieses Dokument gibt Hinweise zu den Standardfunktionsbausteinen der iec61131stdfb Bibliothek. Insbesondere wird auf die Verwendung des Variablentyps \textit{OV\_ANY} und die Verbindungen zu anderen Funktionsblöcken eingegangen. Außerdem wird das Verhalten der mathematischen Funktionsbausteine bei undefinierten Rechenoperationen (z. B. Division durch 0) und bei Werteüberlauf beschrieben.\\






%	- - - - - - - - - - - - - - - - - - - - - - - - - - - - - - - - - - - - - - -
\chapter{Erklärung vorhandener Funktionsblöcke}

In der folgenden Tabelle \ref{TAB_Blocks} sind die vorhandenen Funktionsbausteine nach Verwendung aufgelistet. Nachfolgend wird auf die einzelnen Gruppen kurz eingegangen. Die einzelnen Funktionsblöcke sind weitestgehend selbsterklärend.\\
{
\begin{table}
\centering
\renewcommand{\arraystretch}{1.2}
\parbox{12.75cm}{\caption{\label{TAB_Blocks}Liste der Funktionsblöcke der iec61131stdfb-Bibliothek.}}
\begin{tabular}{|m{2cm}|m{7cm}|>{\raggedright}m{0.75cm} >{\raggedleft}m{1cm}|>{\raggedright}m{0.75cm} >{\raggedleft\arraybackslash}m{1cm}|}
\hline
\textbf{Name}&\textbf{Beschreibung}&\multicolumn{2}{|m{1.75cm}}{\textbf{Eingänge}}&\multicolumn{2}{|>{\arraybackslash}m{1.75cm}|}{\textbf{Ausgänge}}\\
\hline
\multicolumn{6}{|l|}{\textbf{Typumwandlungen}}\\
\hline
\multirow{2}{*}{ANYtoANY}&\multirow{2}{7cm}{Typumwanlung von beliebigem Typ in den durch K spezifizierten (sofern Sinnvoll)}& K & UINT& OUT & ANY\\
 & &IN & ANY& &\\
\hline
\multicolumn{6}{|l|}{\textbf{Grundarithmetik}}\\
\hline
\multirow{2}{*}{ADD}&\multirow{2}{7cm}{Addition zweier Variablen}& IN1 & ANY& OUT & ANY\\
 & &IN2 & ANY& &\\
\hline
\multirow{2}{*}{SUB}&\multirow{2}{7cm}{Subtrahiert IN2 von IN1}& IN1 & ANY& OUT & ANY\\
 & &IN2 & ANY& &\\
\hline
\multirow{2}{*}{MUL}&\multirow{2}{7cm}{Multiplikation zweier Variablen}& IN1 & ANY& OUT & ANY\\
 & &IN2 & ANY& &\\
\hline
\multirow{2}{*}{DIV}&\multirow{2}{7cm}{Dividiert IN1 durch IN2}& IN1 & ANY& OUT & ANY\\
 & &IN2 & ANY& &\\
\hline
\multirow{2}{*}{EXPT}&\multirow{2}{7cm}{Potenziert IN1 hoch IN2}& IN1 & ANY& OUT & ANY\\
 & &IN2 & ANY& &\\
\hline
\multirow{2}{*}{MOD}&\multirow{2}{7cm}{Gibt den Rest der operation IN1 / IN2 an, nur gültig für Integer-Datentypen)}& IN1 & ANY& OUT & ANY\\
 & &IN2 & ANY& &\\
\hline
MOVE & Verschiebt IN nach OUT (reicht durch) & IN & ANY & OUT & ANY\\
\hline

\multicolumn{6}{|l|}{\textbf{Numerische Funktionen}}\\
\hline
ABS & Bildet den Absolutbetrag von IN & IN & ANY & OUT & ANY\\
\hline
SQRT & Bildet die Quadratwurzel von IN & IN & ANY & OUT & ANY\\
\hline
LN & Bildet den natürlichen Logarithmus von IN & IN & ANY & OUT & ANY\\
\hline
LOG & Bildet den Logarithmus zur Basis 10 von IN & IN & ANY & OUT & ANY\\
\hline
EXP & Potenziert zur Basis e (natürliche Exponentiation) & IN & ANY & OUT & ANY\\
\hline
SIN & Bildet den Sinus von IN (in Radian) & IN & ANY & OUT & ANY\\
\hline
COS & Bildet den Kosinus von IN (in Radian) & IN & ANY & OUT & ANY\\
\hline
TAN & Bildet den Tangens von IN (in Radian) & IN & ANY & OUT & ANY\\
\hline
ASIN & Bildet den Arkussinus (in Radian) von IN & IN & ANY & OUT & ANY\\
\hline
ACOS & Bildet den Arkuskosinus (in Radian) von IN & IN & ANY & OUT & ANY\\
\hline
ATAN & Bildet den Arkustangens (in Radian) von IN & IN & ANY & OUT & ANY\\
\hline
\multirow{3}{*}{ATAN2}&\multirow{3}{7cm}{Bildet den Arkustangens (in Radian) von IN1 / IN2 (Winkel des Vektors (IN1 | IN2) zur X-Achse)}& IN1 & ANY& OUT & ANY\\
 & &IN2 & ANY& &\\
 & & & & &\\
\hline
\end{tabular}
\renewcommand{\arraystretch}{1}
\end{table}

	%pagebreak
\begin{table}
\centering
\renewcommand{\arraystretch}{1.2}
\begin{tabular}{|m{2cm}|m{7cm}|>{\raggedright}m{0.75cm} >{\raggedleft}m{1cm}|>{\raggedright}m{0.75cm} >{\raggedleft\arraybackslash}m{1cm}|}
\hline
\multicolumn{6}{|l|}{\textbf{Bitverschiebungen}}\\
\hline
\multirow{2}{*}{SHL}&\multirow{2}{7cm}{Verschiebt die Bitfolge in IN um N Bits nach links. Füllt dabei rechts mit 0en auf.}& IN & UINT & OUT & UINT\\
 & &N & UINT & &\\
\hline
\multirow{2}{*}{SHR}&\multirow{2}{7cm}{Verschiebt die Bitfolge in IN um N Bits nach rechts. Füllt dabei links mit 0en auf.}& IN & UINT & OUT & UINT\\
 & &N & UINT & &\\
\hline
\multirow{3}{*}{ROL}&\multirow{3}{7cm}{Rotiert die Bitfolge in IN um N Bits nach links. Die links überlaufenden Bits werden rechts nachgeschoben.}& IN & UINT & OUT & UINT\\
 & &N & UINT & &\\
 & & & & &\\
\hline
\multirow{3}{*}{ROR}&\multirow{3}{7cm}{Rotiert die Bitfolge in IN um N Bits nach rechts. Die rechts überlaufenden Bits werden links nachgeschoben.}& IN & UINT & OUT & UINT\\
 & &N & UINT & &\\
 & & & & &\\
\hline
\multicolumn{6}{|l|}{\textbf{Bitweise Logik}}\\
\hline
\multirow{2}{*}{AND}&\multirow{2}{7cm}{VerUNDet IN1 und IN2 bitweise.}& IN1 & ANY & OUT & ANY\\
 & &IN2 & ANY & &\\
\hline
\multirow{2}{*}{OR}&\multirow{2}{7cm}{VerODERt IN1 und IN2 bitweise.}& IN1 & ANY & OUT & ANY\\
 & &IN2 & ANY & &\\
\hline
\multirow{2}{*}{XOR}&\multirow{2}{7cm}{Bitweise Exklusiv-ODER Verknüpfung von IN1 und IN2}& IN1 & ANY & OUT & ANY\\
 & &IN2 & ANY & &\\
\hline
NOT & Negiert IN bitweise & IN & ANY & OUT & ANY\\
\hline

\multicolumn{6}{|l|}{\textbf{Auswahlfunktionen}}\\
\hline
\multirow{3}{*}{SEL}&\multirow{3}{7cm}{Reicht IN0 an OUT weiter, wenn G == 0. Andernfalls wird IN1 weitergereicht.}& IN0 & ANY & OUT & ANY\\
 & &IN1 & ANY & &\\
 & & G & BOOL & &\\
\hline
\multirow{2}{*}{MAX}&\multirow{2}{7cm}{Gibt den größeren Wert aus IN1 und IN2 an OUT weiter.}& IN1 & ANY & OUT & ANY\\
 & &IN2 & ANY & &\\
\hline
\multirow{2}{*}{MIN}&\multirow{2}{7cm}{Gibt den kleineren Wert aus IN1 und IN2 an OUT weiter.}& IN1 & ANY & OUT & ANY\\
 & &IN2 & ANY & &\\
\hline
\multirow{3}{*}{LIMIT}&\multirow{3}{7cm}{Begrenzt den Wertebreich von IN auf das Intervall [MN | MX].}& IN & ANY & OUT & ANY\\
 & & MN & ANY & &\\
 & & MX & ANY & &\\
\hline
\multirow{4}{*}{MUX}&\multirow{4}{7cm}{Demultiplext IN1 bis IN8 auf OUT. K gibt die Nummer des Durchgereichten INx an.}& IN1 & ANY & OUT & ANY\\
 & & - & & &\\ 
 & & IN8 & ANY & &\\
 & & K & UINT & &\\
\hline
\end{tabular}
\renewcommand{\arraystretch}{1}
\end{table}
	%pagebreak
\begin{table}
\centering
\renewcommand{\arraystretch}{1.2}
\begin{tabular}{|m{2cm}|m{7cm}|>{\raggedright}m{0.75cm} >{\raggedleft}m{1cm}|>{\raggedright}m{0.75cm} >{\raggedleft\arraybackslash}m{1cm}|}
\hline
\multicolumn{6}{|l|}{\textbf{Vergleichsfunktionen}}\\
\hline
\multirow{2}{*}{GT}&\multirow{2}{7cm}{Ergibt TRUE, wenn IN1 größer als IN2}& IN1 & ANY & OUT & BOOL\\
 & &IN2 & ANY & &\\
\hline
\multirow{2}{*}{GE}&\multirow{2}{7cm}{Ergibt TRUE, wenn IN1 größer oder gleich IN2}& IN1 & ANY & OUT & BOOL\\
 & &IN2 & ANY & &\\
\hline
\multirow{2}{*}{EQ}&\multirow{2}{7cm}{Ergibt TRUE, wenn IN1 gleich IN2}& IN1 & ANY & OUT & BOOL\\
 & &IN2 & ANY & &\\
\hline
\multirow{2}{*}{LE}&\multirow{2}{7cm}{Ergibt TRUE, wenn IN1 kleiner oder gleich IN2}& IN1 & ANY & OUT & BOOL\\
 & &IN2 & ANY & &\\
\hline
\multirow{2}{*}{LT}&\multirow{2}{7cm}{Ergibt TRUE, wenn IN1 kleiner als IN2}& IN1 & ANY & OUT & BOOL\\
 & &IN2 & ANY & &\\
\hline
\multirow{2}{*}{NE}&\multirow{2}{7cm}{Ergibt TRUE, wenn IN1 ungleich IN2}& IN1 & ANY & OUT & BOOL\\
 & &IN2 & ANY & &\\
\hline
\multicolumn{6}{|l|}{\textbf{Zähler}}\\
\hline
\multirow{4}{*}{CTU}&\multirow{4}{7cm}{Zählt CV mit jeder steigenden Flanke in CU um 1 hoch. Wird PV überschritten wird zusätzlich Q auf TRUE gesetzt. Erreicht der Zähler PVmax, wird nicht weiter gezählt. TRUE an R setzt den zähler auf 0 zurück.}& CU & BOOL & Q & BOOL\\
 & &R & BOOL & CV & INT\\
 & & PV & INT & &\\
 & & PVmax & INT & &\\
\hline
\multirow{4}{*}{CTD}&\multirow{4}{7cm}{Zählt CV mit jeder steigenden Flanke in CD um 1 herunter. Wird 0 unterschritten wird zusätzlich Q auf TRUE gesetzt. Erreicht der Zähler PVmin, wird nicht weiter gezählt. TRUE an LD setzt den zähler auf PV zurück.}& CD & BOOL & Q & BOOL\\
 & &LD & BOOL & CV & INT\\
 & & PV & INT & &\\
 & & PVmin & INT & &\\
\hline
\multirow{7}{*}{CTUD}&\multirow{7}{7cm}{Kombiniert CTU und CTD. Steigende Flanken an CU erhöhen CV um 1, an CD verringern CV um 1. QU und QD werden beim Überschreiten von PV bzw. beim unterschreiten von 0 gesetzt. Bei Über- bzw Unterschreiten von PVmax bzw. PVmin wird nicht weiter gezählt. TRUE an R setzt CV auf 0, an LD auf PV zurück.}& CU & BOOL & QU & BOOL\\
 & & CD & BOOL & QD & BOOL\\
 & & R & BOOL & CV & INT\\
 & & LD & BOOL & &\\
 & & PV & INT & &\\
 & & PVmax & INT & &\\
 & & PVmin & INT & &\\
\hline

\multicolumn{6}{|l|}{\textbf{Timer}}\\
\hline
\multirow{2}{*}{TP}&\multirow{2}{7cm}{Steigende Flanke an IN setzt Q für die in PT angegebene Zeitspanne.}& IN & BOOL & Q & BOOL\\
 & &PT & TS(*) & ET & TS(*)\\
\hline
\multirow{2}{*}{TON}&\multirow{2}{7cm}{Einschaltverzögerung}& IN & BOOL & Q & BOOL\\
 & &PT & TS(*) & ET & TS(*)\\
\hline
\multirow{2}{*}{TOFF}&\multirow{2}{7cm}{Ausschaltverzögerung}& IN & BOOL & Q & BOOL\\
 & &PT & TS(*) & ET & TS(*)\\
\hline
\multicolumn{6}{l}{
\begin{footnotesize}
(*) TS: TIME\_SPAN: Variable für Zeitspannen
\end{footnotesize}}
\end{tabular}
\renewcommand{\arraystretch}{1}
\end{table}
	%pagebreak
\begin{table}
\centering
\renewcommand{\arraystretch}{1.2}
\begin{tabular}{|m{2cm}|m{7cm}|>{\raggedright}m{0.5cm} >{\raggedleft}m{1.25cm}|>{\raggedright}m{0.5cm} >{\raggedleft\arraybackslash}m{1.25cm}|}
\hline

\multicolumn{6}{|l|}{\textbf{Bistabile Blöcke}}\\
\hline
\multirow{2}{*}{SR}&\multirow{2}{7cm}{FlipFlop mit Dominanz auf Set}& S1 & BOOL & Q1 & BOOL\\
 & &R & BOOL & &\\
\hline
\multirow{2}{*}{RS}&\multirow{2}{7cm}{FlipFlop mit Dominanz auf Reset}& S & BOOL & Q1 & BOOL\\
 & &R1 & BOOL & &\\
\hline

\multicolumn{6}{|l|}{\textbf{Flankenerkennungen}}\\
\hline
RTRIG & Setzt Q für einen Zyklus auf TRUE, wenn eine steigende Flanke er in CLK erkannt wird. & CLK & BOOL & Q & BOOL\\
\hline
FTRIG & Setzt Q für einen Zyklus auf TRUE, wenn eine fallende Flanke er in CLK erkannt wird. & CLK & BOOL & Q & BOOL\\
\hline
\multicolumn{6}{|l|}{\textbf{Funktionen für Zeichenketten}}\\
\hline
LEN & Gibt die Länge einer Zeichenkette zurück. & IN1 & STRING & OUT & UINT\\
\hline
\multirow{2}{*}{LEFT}&\multirow{2}{7cm}{Gibt L Zeichen von der linken Seite von IN an OUT weiter.}& IN & STRING & OUT & STRING\\
 & & L & UINT & &\\
\hline
\multirow{2}{*}{RIGHT}&\multirow{2}{7cm}{Gibt L Zeichen von der rechten Seite von IN an OUT weiter.}& IN & STRING & OUT & STRING\\
 & & L & UINT & &\\
\hline
\multirow{3}{*}{MID}&\multirow{3}{7cm}{Gibt L Zeichen ab dem P-ten Zeichen von IN an OUT weiter.}& IN & STRING & OUT & STRING\\
 & & L & UINT & &\\
 & & P & UINT & &\\
\hline
\multirow{2}{*}{CONCAT}&\multirow{2}{7cm}{Konkateniert IN1 und IN2.}& IN1 & STRING & OUT & STRING\\
 & & IN2 & STRING & &\\
\hline
\multirow{3}{*}{INSERT}&\multirow{3}{7cm}{Setzt IN2 ab der Position des P-ten Zeichens in IN1 ein}& IN1 & STRING & OUT & STRING\\
 & & IN2 & STRING & &\\
 & & P & UINT & &\\
\hline
\multirow{3}{*}{DELETE}&\multirow{3}{7cm}{Löscht L Zeichen ab dem P-ten Zeichen von IN.}& IN & STRING & OUT & STRING\\
 & & L & UINT & &\\
 & & P & UINT & &\\
\hline
\multirow{4}{*}{DELETE}&\multirow{4}{7cm}{Ersetzt L Zeichen ab der Position des P-ten zeichens in IN1 durch IN2.}& IN1 & STRING & OUT & STRING\\
 & & IN2 & STRING & &\\
 & & P & UINT & &\\
 & & L & UINT & &\\
\hline
\multirow{2}{*}{FIND}&\multirow{2}{7cm}{Gibt die Position des ersten Auftretens von IN2 in IN1 an.}& IN1 & STRING & OUT & UINT\\
 & & IN2 & STRING & &\\
\hline
\end{tabular}
\renewcommand{\arraystretch}{1}
\end{table}
	%Pagebreak
\begin{table}
\centering
\renewcommand{\arraystretch}{1.2}
\begin{tabular}{|m{2cm}|m{7cm}|>{\raggedright}m{0.5cm} >{\raggedleft}m{1.25cm}|>{\raggedright}m{0.5cm} >{\raggedleft\arraybackslash}m{1.25cm}|}
\hline
\multicolumn{6}{|l|}{\textbf{Sonstige Funktionsblöcke}}\\
\hline
\multirow{11}{*}{StateWatch}&\multirow{11}{7cm}{Überwacht den Status der Variable IN. Der aktuelle Status wird in der Variable CurState ausgegeben. In HasState wird angegeben, ob IN überhaupt ein Status-Flag hat. Die Ausgänge CXxx geben die Aktuellen Statusflags wieder. Die Ausgänge HXxx bleiben bei Auftreten so lange gesetzt, bis sie über Reset zurückgesetzt werden. Der Ausgang CurState ist vom Typ UINT, OUT von Typ ANY. Alle anderen Ausgänge sind vom Typ BOOL. Aus Platzgründen sind die Typen rechts nicht angegeben.}& IN & ANY & OUT & ANY\\
 & & & & CurState &\\ 
 & & Reset & BOOL & HasState &\\
 & & & & CBad &\\
 & & & & CGood &\\
 & & & & CQuestionable &\\
 & & & & CUnknown &\\
 & & & & HBad &\\
 & & & & HGood &\\
 & & & & HQuestionable &\\
 & & & & HUnknown &\\
\hline
\end{tabular}
\renewcommand{\arraystretch}{1}
\end{table}
}

Angesehen vom Block \textit{ABS} geben die Blöcke der Numerischen Funktionen ihr Ergebnis als SINGLE oder, wenn der Input DOUBLE ist, als DOUBLE zurück. Eine Ausgabe als Ganzzahl macht hier in den meisten Fällen keinen Sinn. Da die Berechnungen ohnehin mit Fließkommazahlen ausgeführt werden (müssen), würde sich aus Integern auch kein Vorteil in der Rechengeschwindigkeit ergeben.\\
Blöcke der Bereiche Grundarithmetik und numerische Funktionen können Wertebereichsüberschreitungen oder undefinierte Operationen teilweise abfangen. Sie reagieren auf diese Fälle durch setzen des Statusflags \textit{Bad} der Ausgangsvariable. Genaueres hierzu findet sich in Kapitel~\ref{CHAP_Reactions}.\\

Die Bitweise Logik arbeitet innerhalb der ANY-Variablen nur mit UINT, BYTES und BOOLs. Da die Verknüpfungen bitweise erfolgen, sind diese Typen ausreichend. BOOL wird zwar im Speicher wie eine unsigned int-Variable abgelegt, trotzdem wird nur zwischen zwei Werten (TRUE oder FALSE) unterschieden.\\

Auswahlfunktionen sind für alle Datentypen gültig. Bei Minimums- und Maximumsauswahl wird für BOOL'sche Variablen $TRUE > FALSE$ angenommen. Zeichenketten werden mit ov\_string\_compare(...) verglichen. Die selben Vergleichskriterien sind für die Vergleichsfunktionen gültig. Auch hier sind alle Typen einsetzbar. Lediglich bei Vektoren wird nur ihre Länge verglichen (ein Vergleich der einzelnen Elemente würde einen BOOL'schen Vektor als Ausgang erfordern).\\

Die Variable ET der Timer-Blöcke gibt die verstrichene Zeit seit Auslösung an. Ist die über PT eingestellte Zeit erreicht wird jedoch nicht weiter gezählt. TP und TON werden durch eine steigende Flanke ausgelöst, TOFF durch eine fallende.\\
Bei Einschaltverzögerungen wird auf einer fallenden Flanke ET sofort sofort zurückgesetzt. Das gleiche gilt für steigende Flanken am Eingang einer Ausschaltverzögerung. Erreicht ET PT, so wird ein- bzw ausgeschaltet.\\

\vspace{2cm}

Der Block \textit{StateWatch} dient der Überwachung der Status-Flags von ANY-Variablen. Über die Ausgänge CXxx kann der Status der Variable im aktuellen Zyklus ausgelesen werden, zum Beispiel um in folgenden Blöcken auf eine ungültige (CBad == TRUE) Eingabe zu reagieren. Die Ausgänge HXxxx halten einen Aufgetretenen Status fest, bis sie über Reset zurückgesetzt werden. Sie können so als Rückmeldung an den operator dienen.\\

\section{Benutzung des Datentyps OV\_ANY}

Der Datentyp OV\_ANY ist in der Lage jeden im OV-System definierten Datentyp zu repräsentieren. Da die meisten Blöcke mit vielen verschiedenen Typen arbeiten können bietet sich der Einsatz des Typs OV\_ANY hier an. Trotzdem unterstützt nicht jeder Block jeden Typ. Die Ausnahmen sind bereits in der Erklärung der Blöcke dargestellt, daher wird hier aus Wiederholung verzichtet.\\

Um Typensicherheit zu gewährleisten ist eine \glqq Umschaltung\grqq der Datentypen an den Eingängen der Bausteine nicht immer erlaubt. Um zu vermeiden, dass sich während der Laufzeit eines Programms Typenkonflikte ergeben, kann der Datentyp innerhalb der ANY-Struktur am Eingang eines Funktionsbausteins nur geändert werden, wenn der Funktionsbaustein über keine Verbindung zu anderen Funktionsbausteinen (fb\_connection, oder davon abgeleitet) verfügt.\\
Ist der Baustein verbunden, so führt ein Versuch der Änderung des Datentyps zu einer Fehlermeldung \texttt{OV\_ERR\_NOACCESS}. Das bedeutet, dass der Typ, mit dem ein Funktionsbaustein arbeiten soll festgelegt werden muss, bevor der Baustein mit anderen Verbunden wird. Dies kann durch einmalige Zuweisung eines Wertes umgesetzt werden. Bei Instantiierung ist für arithmetische und numerische Funktionsbausteine sowie für Vergleichsoperationen der Typ OV\_SINGLE voreingestellt. Bitweise Logikbausteine sind auf OV\_BOOL eingestellt.\\


	


\section{Umgang mit Überläufen und undefinierten Operationen \label{CHAP_Reactions}}

Bei den mathematischen Funktionen kann es je nach Eingangsparametern zu undefinierten Operationen (z. B. Division durch 0) kommen. Solche Zustände werden erkannt und die Funktionsbausteine reagieren entsprechend.\\

Bei einer Division durch 0 von Realzahlen wird das Ergebnis für positive Dividenden auf \texttt{+INF} (positive Unendlichkeit) und für negative Dividenden auf \texttt{-INF} gesetzt. Wird 0 durch 0 geteilt, so ist das Ergebnis ebenfalls 0. In allen drei Fällen wird der Status der Ergebnisvariablen auf \texttt{OV\_ST\_BAD} gesetzt.\\
Eine Division durch 0 von Integern wird gar nicht erst ausgeführt, da es hierdurch, je nach FB-System zu einem Server-Absturz kommen kann. Das Ergenis wird, analog zu Realzahlen, auf den größtmöglichen bzw. kleinstmöglichen Integer-Wert oder 0 gesetzt. Auch hier wird der Status auf \texttt{OV\_ST\_BAD} gesetzt.\\

Wird versucht, einer numerische Funktion ein Parameter außerhalb des Definitionsbereiches zu übergeben (z. B. $\mathtt{asin(2)}$) so reagiert die Funktion mit der Rückgabe des Wertes \texttt{NaN} (Not a Number) und Setzen des Status \texttt{OV\_ST\_BAD}.\\

Kommt es bei einer Rechenoperation zu einem Werteüber- oder -unterlauf, so wird dieser bei Realzahlen erkannt. Die Ausgangsvariable führt dann den Wert \texttt{HUGE\_VAL} (\texttt{+INF}) bzw. \texttt{-HUGE\_VAL} (\texttt{-INF}). Auch hier wird der Status auf \texttt{OV\_ST\_BAD} gesetzt.\\
Bei einer Rechnung mit Integern kann ein Über- oder Unterlauf nicht erkannt werden. Die überlaufenden Bits werden einfach verworfen. Eine Erkennung der Über- / Unterläufe wäre möglich durch die Berechnung als Fließkommazahl und anschließende Rückkonvertierung. Dies würde jedoch viel Rechenzeit erfordern, so dass die Rechnung nicht mehr schneller wäre, als der Umgang mit Realzahlen.\\

